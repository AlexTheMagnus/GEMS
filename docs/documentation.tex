\documentclass[a4paper,10pt]{article}
\usepackage[utf8]{inputenc} %Codificacion utf-8
\usepackage{graphicx}
\usepackage{enumerate}
\usepackage{fancyhdr}
\usepackage{hyperref}
\usepackage{tikz}     % graphs!
\usepackage{listings} % code!
\usepackage{multirow} % Required for multirows
\usepackage[spanish, activeacute]{babel} %Definir idioma español
% \usepackage[margin=3cm]{geometry}
\usepackage{cite} % para contraer referencias
\usepackage{url}
\makeatletter
\renewenvironment{thebibliography}[1]
     {\section*{\bibname}% <-- this line was changed from \chapter* to \section*
      \@mkboth{\MakeUppercase\bibname}{\MakeUppercase\bibname}%
      \list{\@biblabel{\@arabic\c@enumiv}}%
           {\settowidth\labelwidth{\@biblabel{#1}}%
            \leftmargin\labelwidth
            \advance\leftmargin\labelsep
            \@openbib@code
            \usecounter{enumiv}%
            \let\p@enumiv\@empty
            \renewcommand\theenumiv{\@arabic\c@enumiv}}%
      \sloppy
      \clubpenalty4000
      \@clubpenalty \clubpenalty
      \widowpenalty4000%
      \sfcode`\.\@m}
     {\def\@noitemerr
       {\@latex@warning{Empty `thebibliography' environment}}%
      \endlist}
\makeatother

\usepackage[nottoc,numbib]{tocbibind}
\setlength{\parskip}{1em}
\setlength{\headheight}{15pt}
\hypersetup{
    colorlinks=true,
    linkcolor=blue,
    filecolor=magenta,
    urlcolor=cyan,
}
\pagestyle{fancy}

\lhead{GEMS}
\rfoot{Página \thepage}
\lfoot{Sistemas inteligentes}
\cfoot{}

\newcommand\tab[1][1cm]{\hspace*{#1}}

\title{GEMS, an Expert Managment System.}
\author{Manuel Alejandro Luque León, 46269831H\\Diego Rodríguez Riera, 15404939E}

\begin{document}

\maketitle
\pagebreak
\tableofcontents
\pagebreak

\section{Descripción del problema}
El problema propone la creación de un \textbf{sistema experto para la clasificación de minerales}. Para esto se seguirán las líneas expuestas a continuación:

\textit{Dadas las características de un mineral (por ejemplo, forma, color, dureza,...), el sistema debe indicar el nombre de dicho mineral.  El número mínimo de características a considerar debe ser 5, y el número de minerales diferentes que tenga predefinido el sistema debe ser de 25.}
\pagebreak


\section{Análisis del problema}
\paragraph{Cualquier} otro detalle de análisis, diseño e implementación (poniendo énfasis en las técnicas usadas para controlar el razonamiento) que pueda ser de interés.
\pagebreak


\section{Manual de usuario}
\paragraph{Una sección} dedicada al Manual de Usuario, incluyendo un ejemplo tipo tutorial, paso a paso.


\section{Colaboradores}
Ambos autores hemos trabajado conjuntamente sobre github, Diego Rodríguez Riera se especializó en el diseño y programación del sistema experto y Manuel Alejandro Luque León en la recolecta de información y creación de la base de datos. Finalmente, ambos trabajamos por igual en la creción de la documentación.
\pagebreak

\nocite{*}

\bibliography{biblio.bib}
\bibliographystyle{alpha}
\end{document}
